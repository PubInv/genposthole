\documentclass{article}
\usepackage{graphicx}
\usepackage{amsmath}
\newtheorem{problem}{Problem}
\begin{document}

\title{The Generalized Spherical Rotation Hole-and-Post Cover Problem}
\author{Robert L. Read}

\maketitle

\begin{abstract}
  It is possible to generalize a ball-and-socket mechanical joint
  to allow multiple members to be attached concentrically so that
  each radiates directly fromt he center of the ball and supports
  a limited amount of angular displacement. Because such joints
  a completely concentric, they have the mechanical advantage that
  force is purely compressive or tensile in the members, making them
  tensegrities.  It is not obvious, however, how much angular displacement
  can be supported by a physical joint. In other words, can a given joint
  support a given configuration of members coming in from different directions?
  This paper formalizes this problem mathematically and addresses its
  solution using numerical optimization methods.

\end{abstract}

\section{Introduction}


\section{Problem Statement}
  Define the Generalized Spherical Rotation Hole-and-Post Cover Problem:

  \begin{problem}
  Given a unit radius sphere $A$ with a sequence of $n$ circular holes
  cut out whose centers are given by the unit vectors $C_i$ and
  half-angles (less than $\pi/2$) are given by $A_i$ and
  a unit radius sphere $B$ with a corresponding sequence of $n$ dots or posts
  given by unit vectors $D_i$, is there a rotation about the origin
  of the $A$ that exposes (or covers) each dot or post with the corresponding
  hole? If not, what rotation mimimizes the sum of the squares of the
  distances from the dot to the corresponding circles if they are not
  exposed?
  \end{problem}

  Although this problem could have an algorithmic solution,
  the problem only has three dimensions (the Euler angles of the rotation of
  $A$ to a cover, for example), and therefore is likely very efficiently
  solved by numerical optimazation methods, even without computation of
  a derivative, given an appropriate objective function to minimize.
  Although a proper algorithm would be more elegant, it is probably beyond
  the author without a great deal of thought, and our primary purpose
  is to make our actual robot function properly without moving into
  configurations which cause it to crack itself apart. A numerical
  solution is expected to be adequately efficient for that purpose.
  The second part of the problem statement clearly anticipates a numeric
  solution.
  \section{An Objective Function}

  We need an objective function which takes Euler angles $\alpha,\beta,\gamma$,
  the sphere $A$ and the sphere $B$ and produces a penalty which is the sum of
  the squares of the distance to the circle of the dot on $B$ to the circle on $A$
  for each pair $i$. This will be zero if the dot is actually in the circle (or
  on the section of the sphere defined by the circles.

  From the Euler angles we construct a rotation matrix which can be applied to the
  center of each circle to rotate it. Assume we are in a right-handed frame of
  reference. We choose to think in terms of extrinsic rotations about the fixed
  axes, and arbitrarily choose the  $z_2-x-z_1$ sequence.

  Wikipedia explains how to compute a rotation matrix $M$ for from $\alpha, \beta$ and $\gamma$
  directly. Let $\angle (a,b)$ denote the angle between (unit) vectors $a$ and $b$.
  The distance between two unit vectors is twice the sin of the half angle.

  \[
p(a,b,\theta) =
\begin{cases}
  0 & \text{if}\ \angle(a,b) <= \theta \\
  (2\sin{\frac{\angle(a,b) - \theta}{2}})^2           & \text{if}\  \angle(a,b) > \theta \\
\end{cases}
  \]

\[
f(x) = \sum_{i=1}^{n} p(D_i,M C_i,A_i)
\]

\section{Motivating Test Case}

We are particularly interested in a tetrobot---a robot composed entirely of tetrahedra
which can change their length. Such robots
can crawl or ``ooze'' in motion reminscient of slugs or snails.
In a sense they are similar to tentacles of a mollusc like an octopus.
We seek to use spherical joints that can implement such tetrobots.

Therefore we in particular want to answer the question is a given configurable
acheivable without breaking the joints. An extension of this is: how can we make
all configurations acheivable? Clearly by either allowing extraordinary high angular
displacement (which is mechanically difficult) or by limiting the angular displacement
of the members by limiting the ratio of maximum extent of a member to minimum extent
we can acheive this.

The most obvious case is to model the center of the holes against a regular tetrahelix.







\section{Conclusion}
Write your conclusion here.

\end{document}
